
%\documentclass[secnumarabic, graphics, floatfix, nofootinbib,tightenlines,nobibnotes,aps,prl,12pt]{article}
\documentclass[12pt]{article}

\usepackage[english]{babel}
\usepackage[utf8x]{inputenc}
\usepackage{amsmath}
\usepackage{graphicx}
\usepackage{hyperref}

\hypersetup{
%    bookmarks=true,         % show bookmarks bar?
%    unicode=false,          % non-Latin characters in Acrobat’s bookmarks
%    pdftoolbar=true,        % show Acrobat’s toolbar?
%    pdfmenubar=true,        % show Acrobat’s menu?
%    pdffitwindow=false,     % window fit to page when opened
%    pdfstartview={FitH},    % fits the width of the page to the window
    pdftitle={Efficient Causal Dynamical Triangulations},    % title
    pdfauthor={Adam Getchell},     % author
    pdfsubject={Causal Dynamical Triangulations},   % subject of the document
%    pdfcreator={Creator},   % creator of the document
%    pdfproducer={Producer}, % producer of the document
    pdfkeywords={cdt quantum gravity}, % list of keywords
%    pdfnewwindow=true,      % links in new window
    colorlinks=true,         % false: boxed links; true: colored links, false is default
    linkcolor=blue,          % color of internal links, red is default
    citecolor=blue,        % color of links to bibliography, 'green' is default
%    filecolor=magenta,      % color of file links
    urlcolor=blue             % color of external links, cyan is default
}

\usepackage{cleveref}
\crefname{equation}{equation}{equations}

\usepackage[toc,page]{appendix}

\title{Efficient Causal Dynamical Triangulations}
\author{\textbf{Adam Getchell}\footnote{\href{mailto:acgetchell@ucdavis.edu}{acgetchell@ucdavis.edu}}\\\textit{Department of Physics,
University of California, Davis, CA, 95616}}

\begin{document}
\maketitle

\begin{abstract}
I review constructing piecewise simplicial manifolds using efficient
methods for constructing Delaunay triangulations. I then evaluate the use of
the Metropolis-Hastings algorithm in the Causal Dynamical Triangulations
program. I highlight inefficiencies and propose solutions.
\end{abstract}

\section{Introduction}

\textit{(Why is QG important? Why is QG hard? What is the CDT approach to QG? What are CDTs successes? What are CDT's problems? How does this paper solve one of its problems?)}

A major unsolved problem in physics is reconciling the classical approach of general relativity with quantum field theory. In QFT, the fields operate on a fixed background. In GR, the background (spacetime) itself is a dynamical participant.

The usual perturbative approach of QFT fails for a number of reasons: first, gravity is non-renormalizable. Second, the usual methods of converting a non-renormalizable theory to a renormalizable one, either by adding new fields (Electroweak) or adding new terms (Quantum Chromodynamics), fail. This does not altogther rule out these approaches, but folks have been trying to do this for a long time and no one has succeeded yet. \textit{(This is another Carlip quote, rework or quantify)}

The current proposals to quantize gravity include string theory, which adds infinitely many degrees of freedom, and loop quantum gravity, which quantizes Hilbert space of states in a non-standard way and defines the holonomies of connections as finite quantum objects. \textit{(Discuss holonomy better, or take out reference if too complicated?)} A third approach, asymptotic safety, assumes that nonrenormalizable quantum gravity is just the infrared end of a renormalization group flow, which in turn originates from a nonperturbative UV fixed point. By adjusting a finite number of coupling constants, the parameter space of the critical surface around the UV fixed point can be reached. \textit{(Better discussion of RNG flow needed or warranted, given the topic?)}

Causal Dynamical Triangulations (CDT) is a lattice field theory which defines a nonperturbative quantum field theory of gravity as a sum over spacetime geometries. The lattice spacing parameter introduces a UV cutoff, which allows a systematic search for a fixed point via adjustment of the bare coupling constants. But even if asymptotic safety proves to be an invalid assumption, a lattice theory of quantum gravity is an effective quantum gravity theory, obtained by integrating out all degrees of freedom except for the spin-2 field.

\begin{quote}
This is a hard problem, no one agrees on the answers, and perhaps if we knew why it was hard maybe it wouldn't be hard.

--Steve Carlip
\end{quote}

\end{document}
