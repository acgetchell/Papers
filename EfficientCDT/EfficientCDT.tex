\documentclass[12pt]{article}
\usepackage{amssymb,amsmath,cite,caption,geometry,graphicx,url,hyperref,cleveref}
\addtolength{\textheight}{.8in}
\addtolength{\textwidth}{.9in}
\addtolength{\topmargin}{-.4in}
\addtolength{\evensidemargin}{-.45in}
\addtolength{\oddsidemargin}{-.45in}

\catcode`\@=11

%       This causes equations to be numbered by section

\@addtoreset{equation}{section}
\def\theequation{\arabic{section}.\arabic{equation}}

%       reset section commands

\catcode`\@=11
\def\thesection{\arabic{section}.}
\def\thesubsection{\arabic{subsection}.}
\def\thesubsubsection{\arabic{subsubsection}.}
\def\appendix{\setcounter{section}{0}
        \def\thesection{Appendix.}
        \def\theequation{\Alph{section}.\arabic{equation}}}
\def\section{\@startsection{section}{1}{\z@}{3.5ex plus 1ex minus
   .2ex}{2.3ex plus .2ex}{\large\bf}}

%      reset footnotes

\renewcommand{\thefootnote}{\fnsymbol{footnote}}
\long\def\@makefntext#1{\parindent 0cm\noindent
\hbox to 1em{\hss$^{\@thefnmark}$}#1}

% Different font in captions
\newcommand{\captionfonts}{\small}
% Attribution of images
\newcommand{\source}[1]{\caption*{Source: {#1}} }
\makeatletter  % Allow the use of @ in command names
\long\def\@makecaption#1#2{%
  \vskip\abovecaptionskip
  \sbox\@tempboxa{{\captionfonts #1: #2}}%
  \ifdim \wd\@tempboxa >\hsize
    {\captionfonts #1: #2\par}
  \else
    \hbox to\hsize{\hfil\box\@tempboxa\hfil}%
  \fi
  \vskip\belowcaptionskip}
\makeatother   % Cancel the effect of \makeatletter

\begin{document}
\begin{titlepage}
\vspace{.5in}
\begin{flushright}
June 2018\\  %date
\end{flushright}
\vspace{.3in}

\begin{center}
{\Large\bf
 Efficient Causal Dynamical Triangulations}\\  %title
\vspace{.4in}
{A.~G{\sc etchell}\footnote{\it email: acgetchell@ucdavis.edu}\\
       {\small\it Department of Physics}\\
       {\small\it University of California}\\
       {\small\it Davis, CA 95616}\\{\small\it USA}}\\[2ex]
\end{center}

\vspace{.3in}
\begin{center}
{\large\bf Abstract}
\end{center}
\vspace*{.1ex}
\begin{center}
\begin{minipage}{4.9in}
{\small
I review constructing piecewise simplicial manifolds using efficient
methods for constructing Delaunay triangulations. I then evaluate the use of
the Metropolis-Hastings algorithm in the Causal Dynamical triangulations
program. I highlight inefficiencies and propose solutions.
}
\end{minipage}
\end{center}
\end{titlepage}
\addtocounter{footnote}{-2}

\section{Introduction}

\begin{quote}
  Nevertheless, due to the inneratomic (sic) movements of electrons, atoms would have to radiate
  not only electromagnetic but also gravitational energy, if only in tiny amounts.
  As this is hardly true in nature, it appears that quantum theory would have to modify
  not only Maxwellian electrodynamics, but also the new theory of gravitation.\cite{einstein_volume_nodate}

  --Einstein, 1916 {\it Approximative Integration of the Field Equations of Gravitation, p.209}
\end{quote}

Quantum gravity is, perhaps, the preeminent hard problem\cite{steve_carlip_why_2014} remaining in theoretical physics, and has been worked on for many years\cite{rovelli_notes_2000}.

Although difficult to test experimentally, a quantum theory of gravity appears to be the key to resolving several important
questions, such as the black hole information paradox.\cite{preskill_black_1992} In many cases, the conclusions of the quantum theory reverse the results of the classical theory
(black hole complementarity\cite{almheiri_black_2013}, wormhole no-go theorems \cite{visser_lorentzian_1996}). Therefore, only a quantum theory of gravity will tell us the ultimate fate of life, the universe, and everything (with apologies to the
late Douglas Adams).\cite{adams_life_1997}

Causal Dynamical Triangulations (CDT) \cite{ambjorn_non-perturbative_2000,j._ambjorn_dynamically_2001,loll_discrete_2003,ambjorn_quantum_2013,cooperman_making_2014} is a useful
approach to quantum gravity. It is based on the Regge action\cite{regge_general_1961}, which describes General Relativity on simplicial manifolds similarly to the Einstein-Hilbert
action on differentiable manifolds, and has been independently validated in 3 and 4 dimensions.\cite{kommu_validation_2011}

Using the Metropolis-Hasting algorithm\cite{robert_metropolis-hastings_2015}, one of several Markov Chain Monte Carlo (MCMC) methods, allows
for the analysis of complex distributions in higher dimensions.\cite{grisins_metropolishastings_2014} It is also relatively straight forward to apply to calculations
of the path integral.

However, and this is the central point of this paper, Metropolis-Hastings algorithms suffer from known problems such as exponentially long convergence times to stationary distributions and sensitivity to step size (from 23\% to 70\% is given as a suitable acceptance rate \cite{bedard_optimal_2008,xing_markov_nodate}); both may occur within the context of CDT.

Methods such as slice sampling, Hamiltonian Monte Carlo, and Simulated Annealing are other methods that may be used instead, and which have comparitive advantages over Metropolis algorithms. But each has respective drawbacks:

Slice sampling \cite{neal_slice_2003} adapts to the characteristic of the sample distribution. However, it must be able to sample distributions directly, which is not always possible. It also runs into difficulties at higher dimensions, as it is non-obvious how to obtain ``efficient" samples.

Hamiltonian Monte Carlo (HMC) computes expectations by exploring a continuous parameter space of probability distributions.\cite{betancourt_conceptual_2017}. In certain implementations
it has been show to be extremely fast and efficient\cite{hoffman_no-u-turn_2011}, but it's not necessarily clear how to set this up for the Regge action. Additionally, the parameters may be hard to tune, and it does not handle multimodality well, and ``crumpled" or ``polymer" phases are generic features of Monte Carlo simulations.\cite{koibuchi_phase_2004} Nontheless, I think this is a possiblity worth exploring in a future paper.

Like HMC, simulated annealing requires a global parameter space to optimize.\cite{busetti_simulated_nodate} Implementing this in the context of CDT has not, to my knowledge, been explored.

In this paper, I examine efficiencies in the Causal Dynamical Triangulations approach by:\\[-4ex]
\begin{enumerate}\addtolength{\itemsep}{-1.5ex}
\item Efficiently initialize using Delaunay tetrahedralization;\cite{cgal:eb-18a}  
\item Apply minimal Deterministic Finite Automata, $\epsilon$-machines from Computational Mechanics\cite{shalizi_computational_1999}, to model entropy rate and other calculations;
\item Improve the acceptance rate of Metropolis-Hastings via iteration of von Neumann's procedure
\end{enumerate}
\vspace*{-1ex}

\section{Background}

The Einstein equation describes the curvature of spacetime $R_{\mu\nu}$ in terms of the stress-energy-momentum tensor $T_{\mu\nu}$:

\begin{align}
  R_{\mu\nu}-\frac{1}{2}Rg_{\mu\nu}=8\pi G_{N}T_{\mu\nu}
\end{align}

The Reimann tensor is given by:

\begin{align}
  R_{\sigma\mu\nu}^{\rho}=\partial_{\mu}\Gamma_{\nu\sigma}^{\rho}-\partial_{\nu}\Gamma_{\mu\sigma}^{\rho}+\Gamma_{\mu\lambda}^{\rho}\Gamma_{\nu\sigma}^{\lambda}-\Gamma_{\nu\lambda}^{\rho}\Gamma_{\mu\sigma}^{\lambda}
\end{align}

Where the Affine connection $\Gamma_{\mu\nu}^{\lambda}$ is defined by:

\begin{align}
  \Gamma_{\mu\nu}^{\lambda}=\frac{1}{2}g^{\lambda\sigma}\left(\partial_{\mu}g_{\nu\sigma}+\partial_{\nu}g_{\sigma\mu}-\partial_{\sigma}g_{\mu\nu}\right)
\end{align}

And the (cylindrically symmetric) metric is:

\begin{align}
  g_{\mu\nu}=\left(\begin{array}{cccc}
    e^{2\lambda} & 0 & 0 & 0\\
    0 & -e^{2\left(\nu-\lambda\right)} & 0 & 0\\
    0 & 0 & -e^{2\left(\nu-\lambda\right)} & 0\\
    0 & 0 & 0 & -\frac{r^{2}}{e^{2\lambda}}
    \end{array}\right)
\end{align}
$R^{\rho}_{\sigma\mu\nu}$ can be thought of as encapsulating the intrinsic curvature (see \Cref{parallel-transport-figure}).
\begin{figure}
  \centering
  \includegraphics[width=2in]{Parallel_Transport.pdf}
  \caption{Parallel Transport on a spherical surface by Fred the Oyster, CC BY-SA 4.0, https://commons.wikimedia.org/w/index.php?curid=35124171 \label{parallel-transport-figure}}
\end{figure}

From the Reimann tensor one obtains the Ricci tensor using $R_{\mu\nu}=R^{\rho}_{\mu\rho\nu}$, and likewise the Ricci scalar is $R=R^{\mu}_{\mu}$ using the Einstein summation convention.

Given the Ricci scalar the Einstein-Hilbert action is:

\begin{align}
I_{EH}=\frac{1}{16\pi G_{N}}\int d^{4}x\sqrt{-g}(R-2\Lambda)
\end{align}

Where $G_{N}$ is Newton's Gravitational constant and $\Lambda$ is the cosmological constant.

Extremizing the Einstein-Hilbert action produces the equations of motion.

\begin{align}
  \partial I_{EH} = 0 \rightarrow R_{\mu\nu}-\frac{1}{2}Rg_{\mu\nu}=8\pi G_{N}T_{\mu\nu}
\end{align}

In quantum mechanics, one is interested in the transition probability amplitude $\langle B|T|A\rangle$, which is the conditional probability of being in state $B$ given previously being in state $A$. This is generally computed using the path integral.

\begin{align}
  \langle B|T|A\rangle=\int\mathcal{D}[g]e^{iI_{EH}}
\end{align}

Such path integrals are typically not directly computable, for a number of reasons. Quantum Field Theory uses perturbative summation techniques such as Feynman diagrams, but these require a notion of renormalizability for various infinite divergences, and gravity has been shown to be definitively non-renormalizable.\cite{shomer_pedagogical_2007}

In 1961 Regge developed his calculus replacing smooth differentiable manifolds with simplicial manifolds, obeying the following two properties\\[-4ex]
\begin{enumerate}\addtolength{\itemsep}{-1.5ex}
\item close: every $(n-1)$-dimensional subsimplex of a simplex in the manifold is also in the manifold;  
\item connectivity: two connected $n$-dimensional simplices share one and only one $(n-1)$-dimensional subsimplex;
\end{enumerate}
\vspace*{-1ex}

From here on, simplicial manifolds will be referred to as triangulations. Of special note are Delaunay Triangulations, which are well-behaved simplicial manifolds
with a circumsphere property of member simplices which may be seen intuitively in \Cref{DT}.

\begin{figure}
  \begin{center}
  \includegraphics[width=2in]{DT1.pdf}
   \includegraphics[width=2in]{NDT.pdf}
  \caption{Delaunay triangulation (left) Not a Delaunay triangulation (right) \label{DT}}
  \end{center}
\end{figure}

The discrete version of the Einstein-Hilbert action is the Regge action:

\begin{align}
  I_{R}=\frac{1}{8\pi G_{N}}\left(\sum\limits_{hinges}A_{h}\delta_{h}-\Lambda\sum\limits_{simplices}V_{s}\right)\label{equation:Regge-Action}
\end{align}

And the discrete version of the path integral is (after a Wick rotation to imaginary time):

\begin{align}
  \langle B|T|A\rangle=\sum\limits_{triangulations\ T}\frac{1}{C(T)}e^{-I_{R}(T)} \label{CDT1}
\end{align}

Here, we take a sum over all inequivalent triangulations. In 1991 Pachner\cite{pachner_p.l._1991}, building on Alexander's work in the 1930s\cite{alexander_combinatorial_1930}
showed that elementary operations, now called Pachner moves, could transform a triangulation $T$ to another manifold $T^{\prime}$ homeomorphic to $T$. The set of all inequivalent triangulations could be then be explored via a series of Pachner moves.\cite{gross_elementary_1992}

\Cref{CDT1} takes advantage of the distinctly causal nature of Causal Dynamical Triangulations (along with the well-defined analytic continuation). The triangulations are foliated by hypersurfaces of distinct time. Using this innovation allows an explicit calculation of the CDT action, which has been done for 2-, 3-, and 4-dimensions. The subject of this paper is the 3D action (Equation 35 from \cite{j._ambjorn_dynamically_2001}):

\begin{eqnarray*}
  I_{CDT}^{(3)} &=& 2\pi k\sqrt{\alpha}N_1^{TL} \\
    &+& N_3^{(3,1)}\left[-3k\text{arcsinh}\left(\frac{1}{\sqrt{3}
    \sqrt{4\alpha +1}}\right)-3k\sqrt{\alpha}\text{arccos}\left(\frac{2\alpha+1}
    {4\alpha+1}\right)-\frac{\lambda}{12}\sqrt{3\alpha+1}\right] \\
    &+& N_3^{(2,2)}\left[2k\text{arcsinh}\left(\frac{2\sqrt{2}\sqrt{2\alpha+1}}
    {4\alpha +1}\right)-4k\sqrt{\alpha}\text{arccos}\left(\frac{-1}{4\alpha+1}
    \right)-\frac{\lambda}{12}\sqrt{4\alpha +2}\right]\label{CDT2}
\end{eqnarray*}

Where $\alpha$ is the length of the timelike edges (spacelike edges are length 1), $k=\frac{1}{8\pi G_{N}}$, and $\lambda=k*\Lambda$.

To evaluate \Cref{CDT1}, we use the Metropolis-Hastings algorithm as follows:\\[-4ex]
\begin{enumerate}\addtolength{\itemsep}{-1.5ex}
\item Selection: Pick a Pachner move;  
\item Acceptance: Make that move with a probability of $a=a_1a_2$, where
\end{enumerate}
\vspace*{-1ex}

\begin{align}
  a_{1}=\frac{move[i]}{\sum\limits_{i}move[i]}
\end{align}

\begin{align}
  a_{2}=e^{\Delta I_{CDT}}
\end{align}

Note that we have divided out the measure factor $\frac{1}{C(T)}$ in \Cref{CDT1}, which we didn't know how to evaluate anyway. (It is a combinatorial weight
of the symmetry group of a particular triangulation $T$, for which we'd have to know details of the distribution we are exploring.)

After thermalization\footnote{Empirically derived at present, but another consideration for optimization.}, the Metropolis-Hastings algorithm gives us the distribution of
triangulations for comoputing the path integral. We can then perform measures on these representative ensembles to calculate properties such as spectral dimension.
\cite{j._ambjorn_spectral_nodate,sotiriou_spectral_2011}

\section{Dynamical System}

\section{Methods}

\section{Results}

\section{Conclusion}

\newpage

\vspace{1.5ex}

\bibliographystyle{ieeetr}
\bibliography{EfficientCDT.bib}
\end{document}
