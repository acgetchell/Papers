%\documentclass[secnumarabic, graphics, floatfix, nofootinbib,tightenlines,nobibnotes,aps,prl,12pt]{article}
\documentclass[12pt]{article}

\usepackage[english]{babel}
\usepackage[utf8x]{inputenc}
\usepackage{amsmath}
\usepackage{graphicx}
\usepackage{hyperref}

\hypersetup{
%    bookmarks=true,         % show bookmarks bar?
%    unicode=false,          % non-Latin characters in Acrobat’s bookmarks
%    pdftoolbar=true,        % show Acrobat’s toolbar?
%    pdfmenubar=true,        % show Acrobat’s menu?
%    pdffitwindow=false,     % window fit to page when opened
%    pdfstartview={FitH},    % fits the width of the page to the window
    pdftitle={The Newtonian approximation in Causal Dynamical Triangulations},    % title
    pdfauthor={Adam Getchell},     % author
    pdfsubject={Causal Dynamical Triangulations},   % subject of the document
%    pdfcreator={Creator},   % creator of the document
%    pdfproducer={Producer}, % producer of the document
    pdfkeywords={cdt quantum gravity}, % list of keywords
%    pdfnewwindow=true,      % links in new window
    colorlinks=true,         % false: boxed links; true: colored links, false is default
%    linkcolor=blue,          % color of internal links, red is default
%    citecolor=red,        % color of links to bibliography, 'green' is default
%    filecolor=magenta,      % color of file links
    urlcolor=cyan             % color of external links, cyan is default
}

\usepackage{cleveref}
\crefname{equation}{equation}{equations}

\title{The Newtonian approximation in Causal Dynamical Triangulations}
\author{\textbf{Adam Getchell}\footnote{\href{mailto:acgetchell@ucdavis.edu}{acgetchell@ucdavis.edu}}\\\textit{Department of Physics, University of California, Davis, CA, 95616}}

\begin{document}
\maketitle

\begin{abstract}
I review how to derive Newton's law from the Weyl strut between two Chazy-Curzon particles. I then apply this approach in Causal Dynamical Triangulations, modifying the algorithm to keep two simplicial complexes with curvature (i.e. mass) a fixed distance within each other (modulo regularized deviations) across all time slices. I then examine the results to determine if CDT produces an equivalent Weyl strut, which can then be used to obtain the Newtonian limit.
\end{abstract}

\section{Introduction}

\textit{Why is QG important? Why is QG hard? What is the CDT approach to QG? What are CDTs successes? What are CDT's problems? How does this paper solve one of its problems?}

A major unsolved problem in physics is reconciling the classical approach of General Relativity with quantum field theory. In QFT, the fields operate on a fixed background. In General Relativity, the background (spacetime) itself is a dynamical participant.

The usual perturbative approach of QFT fails for a number of reasons: first, gravity is non-renormalizable. Second, the usual methods of converting a non-renormalizable theory to a renormalizable one, either by adding new fields (Electroweak) or adding new terms (QCD), fail. This does not altogther rule out these approaches, but folks have been trying to do this for a long time and no one has succeeded yet.

The current proposals to quantize gravity include string theory, which adds infinitely many degrees of freedom, and loop quantum gravity, which quantizes Hilbert space of states in a non-standard way and defines the holonomies of connections as finite quantum objects. A third approach, asymptotic safety, assumes that nonrenormalizable quantum gravity is just the infrared end of a renormalization group flow, which in turn originates from a nonperturbative UV fixed point. By adjusting a finite number of coupling constants, the parameter space of the critical surface around the UV fixed point can be reached.

CDT is a lattice field theory which defines a nonperturbative quantum field theory of gravity as a sum over spacetime geometries. The lattice spacing parameter introduces a UV cutoff, which allows a systematic search for a fixed point via adjustment of the bare coupling constants. But even if asymptotic safety proves to be an invalid assumption, a lattice theory of quantum gravity is an effective quantum gravity theory, obtained by integrating out all degrees of freedom except for the spin-2 field. 

Causal Dynamical Triangulations \cite{ambjorn_geometry_1996,cdt,ambjorn_nonperturbative_2012} is a promising approach to the problems of quantum gravity. Since the 1930's \cite{rovelli_notes_2000} attempts have been made to unify quantum mechanics with general relativity. (This is a hard problem, no one agrees on the answers, and perhaps if we knew why it was hard maybe it wouldn't be hard.) The underlying difficulties are that observables in general relativity are necessarily non-local, making it difficult to write down a theory that extracts observable results.

Causal Dynamical Triangulations uses the path integral approach, and has had notable successes \cite{kommu_validation_2011}. However, a difficulty is taking and extracting data that has physical meaning. (One cannot identify points in a path integral, nor talk about functions of a point.)

Field correlator in matter-coupled quantum gravity

A fundamental question is, does Causal Dynamical Triangulations have physical meaning? Attempts have been made before to relate CDT to the semi-classical limit \cite{ambjorn_semiclassical_2011,ambjorn_semiclassical}, but not everyone is convinced.

This paper attempts to answer this question by directly finding the Newtonian approximation in Causal Dynamical Triangulations.

\section{Newton's Law of Gravitation from General Relativity}
\label{sec:newtons-law}

\textit{What is elementary flatness? What is a conical singularity? What is intrinsic curvature? What is extrinsic curvature? Why does a conical singularity give us Newton's law?}

Starting from the cylindrically symmetric (Weyl) vacuum metric \cite{synge_relativity}

\begin{equation}
	ds^{2}=e^{2\lambda}dt^{2}-e^{2\left(\nu-\lambda\right)}\left(dr^{2}+dz^{2}\right)-r^{2}e^{-2\lambda}d\phi^{2}
	\label{eq:weyl-vacuum-metric}
\end{equation}
where $\lambda$ and $\nu$ are both functions of $r$ and $z$ we find that

\begin{equation}
\partial^{2}_{r}\lambda+\frac{1}{r}\partial_{r}\lambda+\partial^{2}_{z}\lambda=\nabla^2\lambda(r,z)=0
\label{eq:laplace-r-z}
\end{equation}

\begin{equation}
\nu=\int r[\left(\left(\partial_{r}\lambda\right)^{2}-\left(\partial_{z}\lambda\right)^{2}\right)dr+\left(2\partial_{r}\lambda\partial_{z}\lambda\right)dz].
\label{eq:nu}
\end{equation}

The solutions must satisfy \Cref{eq:laplace-r-z,eq:nu}. A particular solution corresponding to two objects (given by Curzon in 1924 \cite{curzon1924} ) is

\begin{equation}
\lambda_0=-\frac{\mu_1}{r_1}-\frac{\mu_2}{r_2}
\label{eq:lambda-0}
\end{equation}

\begin{equation}
	\label{eq:nu-0}
	\nu_0=\frac{1}{2}\frac{\mu_{1}^{2}r^2}{r_{1}^{4}}-\frac{1}{2}\frac{\mu_{2}^{2}r^2}{r_{2}^{4}}+\frac{2\mu_1\mu_2}{(z-z_2)^2}\left[\frac{r^2+(z-z_1)(z-z_2)}{r_{1}r_{2}}-1\right]
\end{equation}
where $z_1$ and $z_2$ correspond to the positions on the z-axis for the two objects, $\mu_1$ and $\mu_2$ are length parameters, and

\begin{equation}
r_1=\sqrt{r^2+(z-z_1)^2}
\label{eq:r_1}
\end{equation}

\begin{equation}
r_2=\sqrt{r^2+(z-z_2)^2}.
\label{eq:r_2}
\end{equation}

Up to this point we have been assuming spacetime is truly flat. We check this assumption via the condition of elementary flatness: the ratio of the circumference to the radius is equal to $2\pi$. 

To do this we integrate in the $\hat{\phi}$ direction at
some $r$ and then divide by $r$. This gives

\begin{equation}
  \label{eq:phi-hat-length}
  C=\int ds =
  \int_0^{2\pi}\sqrt{r^2e^{-2\lambda}d\phi^2}={2\pi re^{-\lambda}}.
\end{equation}

Then the condition that $\frac{C}{r}=2\pi$ holds provided that

\begin{equation}
	\label{eq:lambda-elem-flat}
	\lambda(0,z)\rightarrow 0.
\end{equation} 

But \Cref{eq:lambda-0} contradicts \Cref{eq:lambda-elem-flat} and $\frac{C}{r}$ is not at all well-defined as $r\rightarrow 0$. Indeed, Einstein and Rosen \cite{einstein-rosen-1936} first noted that the Weyl metric cannot be a purely vacuum solution, and that there must be a strut on the z-axis.

Now to the salient point: we can use this strut to our advantage by obtaining $T_{zz}$ and thence the Newtonian gravitational interaction via

\begin{equation}
	\label{eq:F_z}
	F_{z}=\int T_{zz}d\sigma
\end{equation}
as was done by Katz in 1967 \cite{katz1967derivation}. We will use a different approach, however, from either Katz or more recent literature \cite{letelier_superposition_1997}.

Taking the parallel transport of a vector around the strut, we obtain

TODO

Using the appropriate connections we obtain

TODO

Now we can just read off the value of $G_{zz}$ and thence $T_{zz}$ to get

TODO

\section{Geometry}

\textit{Discuss dynamical triangulations. Discuss Regge calculus. Discuss getting the Einstein tensor in Regge calculus. Discuss getting mass from CDT insertions. Discuss ergodic moves.}

In order to translate the concepts of General Relativity to discrete form, a simplicial manifold is used. As implied by their name, simplicial manifolds combine the properties of simplices -- a generalization of triangulations -- with those of manifolds. This allows the useful machinery of differential forms to be carried over to the discrete realm.

The triangulations in Causal Dynamical Triangulations refers to the use of $d$-simplices to construct a spacetime lattice. In general, a $d$-dimensional simplex has $d+1$ points, which are also referred to as $0$-simplices. For a $d$-dimensional simplex there are $\binom{d+1}{k+1}$ $k$-dimensional faces, or sub-simplices.

A simplicial manifold \textit{T} has two defining properties \cite{cgal:eb-12b}:

\begin{enumerate}
  \item Any face of a simplex in \textit{T} is a simplex in \textit{T}
  \item Two simplices in \textit{T} are either disjoint or share a common face
\end{enumerate}

Causal refers to the fact that the triangulations generally span two adjacent timeslices (some simplices -- referred to as spacelike -- do not span timeslices). Using a notation $\{k,n\}$ where $k$ is the number of points on the higher timelike slice and $n$ is the number of points in the lower timelike slice, we summarize simplex geometry in \Cref{table:simplices}. This will be useful in the discussion of ergodic moves in \Cref{ergodic}.

\begin{table}
\centering
\begin{tabular}{|l|c|c|c|c|c|c|c|}
\hline
Name & Dim & 0-faces & 1-faces & 2-faces & 3-faces & 4-faces & Causal Structure \\
\hline
\hline
Vertex & 0 & 1 & & & & & \\
Edge & 1 & 2 & 1 & & & & \{1,1\} \\
Triangle & 2 & 3 & 3 & 1 & & & \{2,1\} \{1,2\}\\
Tetrahedron & 3 & 4 & 6 & 4 & 1 & & \{3,1\} \{2,2\} \{1,3\} \\
Pentatope & 4 & 5 & 10 & 10 & 5 & 1 & \{4,1\} \{3,2\} \{2,3\} \{1,4\} \\
\hline
\end{tabular}
\caption[Simplex types]{Types and causal structures of simplices}
\label{table:simplices}
\end{table}

The general idea behind Causal Dynamical Triangulations is

TODO

To do this, we start from the results of Regge Calculus \cite{regge}

TODO

\subsection{Mass and the Einstein tensor}

Using Barrett \cite{barrett_1986}, we can derive the Einstein tensor in Regge Calculus as follows

TODO

In order to introduce mass, we 

TODO

\subsection{Ergodic moves}
\label[subsection]{ergodic}

\subsection{Newtonian gravity in CDT}

\section{Notes on Implementation}

\textit{Discuss Computational Geometry. Discuss the CDT algorithm. Discuss the algorithm for getting mass. Discuss algorithm for getting stress-energy. Discuss other geometrical algorithms.}

These ideas are implemented using CGAL \cite{cgal}, a time-tested library of geometric algorithms in continuous development since 1995.

TODO

\section{Results}

\textit{Discuss any preliminary results.}

\section{Conclusion}

\textit{Discuss further work. Thank colleagues.}

\bibliographystyle{ieeetr}
\bibliography{cdt-newtonian-limit-biblio}

\end{document}