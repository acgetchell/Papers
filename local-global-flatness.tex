\documentclass{article}
% \usepackage{times}
\usepackage{mathptmx}
\usepackage[T1]{fontenc}
\usepackage[latin9]{inputenc}
\usepackage{calc}
\usepackage{units}
\usepackage{amsmath}
\usepackage{amssymb}
\usepackage{graphicx}
\usepackage{subdepth}
\usepackage{cite}
\usepackage{url}
\usepackage{notoccite}
\usepackage{nag}
\title{Local and Global Flatness}
\author{Adam Getchell}
\date{}
\begin{document}
\maketitle


Given the metric on a cone "induced from the plane":

\begin{equation}
\begin{array}{rcl} ds^{2}=dr^{2}+r^{2}d\theta^{2} & \mbox{with} &  \begin{array}{cc}0\leqslant r\leqslant\infty, & \beta\leqslant\theta < 2\pi

\end{array}
\end{array}
\end{equation}

Using the Christoffel connections:
\begin{equation}
\Gamma_{\mu\nu}^{\lambda}=\frac{1}{2}g^{\lambda\sigma}\left(\partial_{\mu}g_{\nu\sigma}+\partial_{\nu}g_{\sigma\mu}-\partial_{\sigma}g_{\mu\nu}\right)
\end{equation}

We find:
\begin{equation}
\begin{array}{l}
\Gamma^{r}_{\theta\theta}=-r\\
\Gamma^{\theta}_{r\theta}=\Gamma^{\theta}_{\theta r}=\frac{1}{r}\\
\end{array}
\end{equation}

To parallel transport around the curve from $\theta=\beta$ to $\theta=2\pi$ we set the directional covariant derivative equal to zero along the curve:
\begin{equation}
\begin{array}{rcl} \frac{D}{d\lambda}=\frac{dx^{\mu}}{d\lambda}\nabla_{\mu}=0 & \mbox{along} & x^{\mu}\left(\lambda\right)
\end{array}
\end{equation}

Since we are parallel transporting a vector, this reduces to:
\begin{equation}
\nabla_{\alpha}V^{\beta}=0
\end{equation}

We are going around the $\theta$ axis at $r=r_{0}$, so we have:
\begin{equation}
\begin{array}{l}
\partial_{\theta}V^{r}+\Gamma^{r}_{\theta\theta}V^{\theta}=0 \\
\partial_{\theta}V^{\theta}+\Gamma^{\theta}_{\theta r}V^{r}=0
\end{array}
\end{equation}

Plugging in our connections we have:
\begin{equation}
\partial_{\theta}V^{r}-rV^{\theta}=0 \label{eq:v-theta}
\end{equation}
\begin{equation}
\partial_{\theta}V^{\theta}+\frac{1}{r}V^{r}=0 \label{eq:v-r}
\end{equation}

Differentiating Equation (\ref{eq:v-theta}) with respect to $\theta$ we get:

\begin{equation}
\partial_{\theta}V^{\theta}=\frac{1}{r}\partial^{2}_{\theta}V^{r} \label{eq:d-theta-v-theta}
\end{equation}
We can plug Equation (\ref{eq:d-theta-v-theta}) into Equation (\ref{eq:v-r}) to get:

\begin{equation}
\partial^{2}_{\theta}V^{r}+V^{r}=0
\end{equation}

For which the solution is:

\begin{equation}
V^{r}=A\cos(\theta)+B\sin(\theta)
\end{equation}

Likewise,

\begin{equation}
V^{\theta}=\frac{1}{r}\left(-A\cos(\theta)+B\sin(\theta)\right)
\end{equation}

Now, at $\theta=\beta$ then $V=0\hat{e_{r}}+1\hat{e_{\theta}}$ so we have:

\begin{equation}
V^{r}=0=A\cos(\beta)+B\sin(\beta)\rightarrow A=-B\tan(\beta)
\end{equation}

\begin{equation}
V^{\theta}=\frac{1}{r_{0}}\left(-A\cos(\beta)+B\sin(\beta)\right)\rightarrow B=\frac{r_{0}}{\tan(\beta)\sin(\beta)+\cos(\beta)}
\end{equation}

Which gives us (after simplifying trigonometry):

\begin{equation}
\begin{array}{l}
A=-r_{0}\sin(\beta) \\
B=r_{0}\cos(\beta)
\end{array}
\end{equation}

And the expression for $V^{\theta}$ is:

\begin{equation}
V^{\theta}=\left(\sin(\theta)\sin(\beta)+\cos(\theta)\cos(\beta)\right)\hat{e}_{\theta}
\end{equation}

Which gives, finally, that:

\begin{equation}
V^{\theta}\left(\theta=2\pi\right)=\cos(\beta)\hat{e}_{\theta}
\end{equation}

As the first sanity check, we note that if $\beta=0$ then $V^{\theta}=\hat{e}_{\theta}$ as expected.

For the second sanity check, we note that $V_{\theta}V^{\theta}=1$ for any value of $\beta$ or $\theta$, that is, the length of the vector after being parallel transported is unchanged.
\end{document}
