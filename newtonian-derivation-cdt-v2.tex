\documentclass{article}
% \usepackage{times}
\usepackage{mathptmx}
\usepackage[T1]{fontenc}
\usepackage[latin9]{inputenc}
\usepackage{calc}
\usepackage{units}
\usepackage{amsmath}
\usepackage{amssymb}
\usepackage{graphicx}
\usepackage{subdepth}
\usepackage{cite}
\usepackage{url}
\usepackage{notoccite}
\usepackage{nag}
\title{Newtonian approximation in Causal Dynamical Triangulations}
\author{Adam Getchell}
\date{}
\begin{document}
\maketitle
\tableofcontents

\section{Motivation}

\subsection{Newton's Law of Gravitation from General Relativity}

Starting from the most general cylindrically symmetric (Weyl) metric \cite{synge_relativity}:

\begin{equation}
ds^{2}=e^{2\lambda}dt^{2}-e^{2\left(\nu-\lambda\right)}\left(dr^{2}+dz^{2}\right)-r^{2}e^{-2\lambda}d\phi^{2}\label{eq:weyl-metric}
\end{equation}

\begin{equation}
g_{\mu\nu}=\left(\begin{array}{cccc}
e^{2\lambda}dt^{2} & 0 & 0 & 0\\
0 & -e^{2\left(\nu-\lambda\right)}dr^{2} & 0 & 0\\
0 & 0 & -e^{2\left(\nu-\lambda\right)}dz^{2} & 0\\
0 & 0 & 0 & -\frac{r^{2}}{e^{2\lambda}}d\phi^{2}
\end{array}\right)\label{eq:general-axisymmetric-static-matrix-metric}
\end{equation}
The definition of the Christoffel connection is: \cite{carroll2003spacetime} 
\begin{equation}
\Gamma_{\mu\nu}^{\lambda}=\frac{1}{2}g^{\lambda\sigma}\left(\partial_{\mu}g_{\nu\sigma}+\partial_{\nu}g_{\sigma\mu}-\partial_{\sigma}g_{\mu\nu}\right)
\end{equation}
With the assumption of zero torsion:
\begin{equation}
\Gamma_{\mu\nu}^{\lambda}=\Gamma_{\nu\mu}^{\lambda}
\end{equation}
The non-zero Christoffel connections are:
\begin{equation}
\begin{array}{l}
\Gamma^{t}_{tr}=\partial_{r}\lambda\\
\Gamma^{t}_{tz}=\partial_{z}\lambda\\
\Gamma^{r}_{tt}=e^{4\lambda-2\nu}\partial_{r}\lambda\\
\Gamma^{r}_{rr}=\partial_{r}\nu-\partial_{r}\lambda\\
\Gamma^{r}_{rz}=\partial_{z}\nu-\partial_{z}\lambda\\
\Gamma^{r}_{zz}=\partial_{z}\lambda-\partial_{z}\nu\\
\Gamma^{r}_{\phi\phi}=re^{-2\nu}\left(r\partial_{r}\lambda-1\right)\\
\Gamma^{z}_{tt}=e^{4\lambda-2\nu}\partial_{z}\lambda\\
\Gamma^{z}_{rr}=\partial_{z}\lambda-\partial_{z}\nu\\
\Gamma^{z}_{rz}=\partial_{r}\nu-\partial_{r}\lambda\\ 
\Gamma^{z}_{zz}=\partial_{r}\nu-\partial_{r}\lambda\\
\Gamma^{z}_{\phi\phi}=r^{2}e^{-2\nu}\partial_{z}\lambda\\
\Gamma^{\phi}_{r\phi}=\frac{1}{r}-\partial_{r}\lambda\\
\Gamma^{\phi}_{z\phi}=-\partial_{z}\lambda\\
\end{array}
\end{equation}
The components of the Riemann tensor are given by:
\begin{equation}
R_{\sigma\mu\nu}^{\rho}=\partial_{\mu}\Gamma_{\nu\sigma}^{\rho}-\partial_{\nu}\Gamma_{\mu\sigma}^{\rho}+\Gamma_{\mu\lambda}^{\rho}\Gamma_{\nu\sigma}^{\lambda}-\Gamma_{\nu\lambda}^{\rho}\Gamma_{\mu\sigma}^{\lambda}
\end{equation}
Using the properties of the Riemann tensor:
\begin{equation}
\begin{array}{l}
R_{\rho\sigma\mu\nu}=-R_{\rho\sigma\nu\mu}\\
R_{\rho\sigma\mu\nu}=-R_{\sigma\rho\mu\nu}\\
R_{\rho\sigma\mu\nu}=R_{\mu\nu\rho\sigma}\\
R_{\rho[\sigma\mu\nu]}=0\\
\end{array}
\end{equation}
The non-zero components of the Riemann tensor are:
\begin{equation}
\begin{array}{l}
R^{t}_{rtr}=-\partial^{2}_{r}\lambda+\left(\partial_{z}\lambda\right)^{2}-2\left(\partial_{r}\lambda\right)^{2}+\partial_{r}\lambda\partial_{r}\nu-\partial_{z}\lambda\partial_{z}\nu\\
R^{t}_{rtz}=-\partial_{r}\partial_{z}\lambda-3\partial_{r}\lambda\partial_{z}\lambda+\partial_{r}\lambda\partial_{z}\nu+\partial_{r}\nu\partial_{z}\lambda\\
R^{t}_{ztz}=-\partial^{2}_{z}\lambda-2\left(\partial_{z}\lambda\right)^{2}+\left(\partial_{r}\lambda\right)^{2}-\partial_{r}\lambda\partial_{r}\nu+\partial_{z}\lambda\partial_{z}\nu\\
R^{t}_{\phi t\phi}=re^{-2\nu}\left(r\left(\partial_{r}\lambda\right)^{2}-\partial_{r}\lambda+r\left(\partial_{z}\lambda\right)^{2}\right)\\
R^{r}_{zrz}=\partial^{2}_{r}\lambda-\partial^{2}_{r}\nu+\partial^{2}_{z}\lambda-\partial^{2}_{z}\nu\\
R^{z}_{\phi z\phi}=re^{-2\nu}\left(r\partial^{2}_{z}\lambda-r\partial_{z}\lambda\partial_{z}\nu+r\partial_{r}\lambda\partial_{r}\nu-r\left(\partial_{r}\lambda\right)^{2}+\partial_{r}\lambda-\partial_{r}\nu\right)\\
R^{z}_{\phi\phi r}=re^{-2\nu}\left(-r\partial_{r}\partial_{z}\lambda+r\partial_{r}\nu\partial_{z}\lambda-r\partial_{r}\lambda\partial_{z}\lambda+r\partial_{r}\lambda\partial_{z}\nu-\partial_{z}\nu\right)\\
R^{\phi}_{r\phi r}=\partial^{2}_{r}\lambda+\frac{1}{r}\partial_{r}\nu-\partial_{r}\lambda\partial_{r}\nu-\left(\partial_{z}\lambda\right)^{2}+\partial_{z}\lambda\partial_{z}\nu+\frac{1}{r}\partial_{r}\lambda\\
\end{array}
\end{equation}
The Ricci tensor is given by:
\begin{equation}
R_{\mu\nu}=R_{\mu\lambda\nu}^{\lambda}
\end{equation}
The non-zero components of the Ricci tensor are:
\begin{equation}
\begin{array}{l}
R_{tt}=\frac{e^{4\lambda-2\nu}}{r}\left(r\partial^{2}_{r}\lambda+r\partial^{2}_{z}\lambda+\partial_{r}\lambda\right)\\
R_{rr}=\partial^{2}_{r}\lambda-\partial^{2}_{r}\nu+\partial^{2}_{z}\lambda-\partial^{2}_{z}\nu-2\left(\partial_{r}\lambda\right)^{2}+\frac{1}{r}\partial_{r}\lambda+\frac{1}{r}\partial_{r}\nu\\
R_{rz}=\frac{1}{r}\partial_{z}\nu-2\partial_{r}\lambda\partial_{z}\lambda\\
R_{zz}=\partial^{2}_{r}\lambda-\partial^{2}_{r}\nu+\partial^{2}_{z}\lambda-\partial^{2}_{z}\nu-2\left(\partial_{z}\lambda\right)^{2}+\frac{1}{r}\partial_{r}\lambda-\frac{1}{r}\partial_{r}\nu\\
R_{\phi\phi}=re^{-2\nu}\left(r\partial^{2}_{r}\lambda+r\partial^{2}_{z}\lambda+\partial_{r}\lambda\right)
\end{array}\label{eq:ricci-tensor-components}
\end{equation}
Einstein's equation in a vacuum is:
\begin{equation}
R_{\mu\nu}=0\label{eq:vacuum-solutions}
\end{equation}
Applying this complete set of relations to Equation (\ref{eq:ricci-tensor-components}) gives the following:
\begin{equation}
\partial^{2}_{r}\lambda+\frac{1}{r}\partial_{r}\lambda+\partial^{2}_{z}\lambda=0\label{eq:laplace}
\end{equation}
\begin{equation}
\partial_{r}\nu=r\left(\partial^{2}_{r}\nu+\partial^{2}_{z}\nu+2\left(\partial_{r}\lambda\right)^{2}\right)\label{eq:R_rr=0}
\end{equation}
\begin{equation}
\partial_{z}\nu=2r\partial_{r}\lambda\partial_{z}\lambda\label{eq:nu_z}
\end{equation}
\begin{equation}
\partial^{2}_{r}\nu+\partial^{2}_{z}\nu+\left(\partial_{r}\lambda\right)^{2}+\left(\partial_{z}\lambda\right)^{2}=0\label{eq:R_phiphi=0}
\end{equation}
Equation (\ref{eq:laplace}) is the two-dimensional Laplace equation in cylindrical coordinates, for which the known solutions are:
\begin{equation}
\lambda (r,z)=\sum^{\infty}_{m=0}[A_{m}J_{m}\left(kr\right)+B_{m}N_{m}\left(kr\right)][C_{m}\sinh(kr)+D_{m}\cosh(kr)]\label{eq:laplace_sol}
\end{equation}
Plugging Equation (\ref{eq:R_phiphi=0}) into Equation (\ref{eq:R_rr=0}) gives:
\begin{equation}
\partial_{r}\nu=r\left(\left(\partial_{r}\lambda\right)^{2}-\left(\partial_{z}\lambda\right)^{2}\right)\label{eq:nu_r}
\end{equation}
Using Equations (\ref{eq:nu_z}), (\ref{eq:laplace_sol}) and (\ref{eq:nu_r}) we find solutions for $\nu$ given by:
\begin{equation}
\nu=\int r[\left(\left(\partial_{r}\lambda\right)^{2}-\left(\partial_{z}\lambda\right)^{2}\right)dr+\left(2\partial_{r}\lambda\partial_{z}\lambda\right)dz]
\end{equation}
However, before we can consider this to be a complete solution we must consider elementary flatness. This condition requires that, for any infinitesimal spacelike circle, the ratio of circumference to radius is 2$\pi$. The most likely place to run into issues is along the z-axis, for which $r=0$. Looking back at Equation (\ref{eq:weyl-metric}) we see that the necessary condition is:
\begin{equation}
\begin{array}{rcl}
\nu=0 & \mbox{for} & r=0
\end{array}
\end{equation}
\bibliographystyle{ieeetr}
\bibliography{cdt-newtonian-limit-biblio}


\end{document}
